\documentclass[12pt]{article}
\usepackage[usenames,dvipsnames]{color}
\usepackage{listings}
\usepackage{graphicx}
\usepackage{fancyhdr}
\usepackage{framed}
\usepackage[T1]{fontenc}
\usepackage[toc,page]{appendix}
\usepackage[utf8]{inputenc}
\usepackage[brazil]{babel}
\usepackage{fancyvrb}
\usepackage[hmargin=2cm,vmargin=2cm]{geometry}
\usepackage{lastpage}
\usepackage{makeidx}
\pagestyle{fancy}

% cabecalho e rodapé
\setlength{\headheight}{120pt}
\setlength{\textheight}{550pt}
\renewcommand{\headrulewidth}{0pt}
\lhead{\includegraphics[scale=0.03]{brasao.png}}
\rhead{\includegraphics[scale=0.4]{logo-pnud.png}}
\cfoot{\textbf{\ProjectCode\ - Inovando a democracia participativa}}
\rfoot{\thepage}

\hyphenation{par-ti-ci-pa-ção}
\bibliographystyle{ieeetr}

% definições sobre o autor e o produto
\newcommand{\MyName}{Renato Fabbri}
\newcommand{\MySurnameForename}{Fabbri, Renato}
\newcommand{\SupervisorName}{Gabriella Vieira Oliveira Gonçalves}
\newcommand{\MyEmail}{renato.fabbri@gmail.com}
\newcommand{\ContractNumber}{2013/000566}
\newcommand{\ContractYear}{2014}
\newcommand{\ProjectCode}{Projeto BRA/12/018}
\newcommand{\NomeSecretaria}{Secretaria Geral da Presidência da República}
\newcommand{\SiglaSecretaria}{SG/PR}
\newcommand{\ProductNumber}{03}
\newcommand{\ProductTitle}{Ferramentas assistidas de categorização de conteúdo}
\newcommand{\ProductSubtitle}{Com base em Processamento de Linguagem Natural e de Redes Complexas, adaptadas para o ambiente do portal de participação}
\newcommand{\ProductDescription}{"Ferramentas assistidas de categorização de conteúdo: Com base em Processamento de Linguagem Natural e de Redes Complexas, adaptadas para o ambiente do portal de participação."
}
\newcommand{\ProductValue}{R\$ 10,800 (dez mil e oitocentos reais)}
\newcommand{\ObjetoContratacao}{
Aporte de conhecimentos e tecnologias para especificação de vocabulário e ferramentas assistidas que utilizam processamento de linguagem natural e análise de redes complexas para o conteúdo do portal da participação social.
}
\newcommand{\DataEntrega}{27 Julho de 2014}
\newcommand{\PalavrasChave}{reconhecimento de padrões, redes complexas, processamento de linguagem natural, participação social}

% lista de abreviações
\makeindex

\begin{document}

\input{folhaderosto.tex}
\input{folhadeaprovacao.tex}
\input{folhadeidentificacao.tex}
\tableofcontents
\newpage

\begin{abstract}
Este documento descreve procedimentos selecionados para categorização de conteúdo do portal federal da participação social (Participa.br). O produto relacionado no termo de referência desta consultoria preve somente propostas de especificações e códigos. Dado o aspecto prático do trabalho, estão descritas o contexto e possibilidade consideradas, asism como implementações e códigos operantes. Parte deste trabalho é acessível online via http, como os scripts no IPython Notebook e o endpoint SparQL que serve os dados do Participa.br com critérios semânticos.\\

{\bf Palavras-chave:} \PalavrasChave.
\end{abstract}
\newpage

\section{Introdução}
\subsection{Contexto e importância da consultoria}
Em confluência com o portal federal de participação social (Participa.br), e o Plano Nacinal de Participação Social (PNPS), esta consultoria propõe métodos de classificação e priorização de conteúdo,  e formas de auto-regulação para o portal. O presente produto apresenta uma seleção de métodos para classificação de conteúdo. Dadas a pertinência para o contexto participativo e a simplicidade, são apresentadas a classificação via 1) conectividade dos autors e 2) características dos documentos.
\subsection{Contexto e importância do Produto}
\begin{itemize}
    \item Este Produto, através da classificação de conteúdos, visa 1) facilitar a assimilação das informações pelos participantes; 2) explicitar propriedades do sistema considerado; 3) permitir observação de conteúdos produzidos por nichos ou características em comum.
    \item São esperadas a incorporação destes métodos no funcionamento do Participa.br e pelos participantes.
    \item A especialização conectiva dos agentes sociais, e do texto produzido por indivíduos e grupos, é um fenômeno plenamente reconhecido. O aproveitamento destas diferenciações é uma realidade, mesmo ainda restrito a empresas e acadêmicos. A entrega prática destes conhecimentos ao poder federal e à sociedade capacita a democracia participativa.
\end{itemize}

\section{Desenvolvimento}

O Participa.br é a Plataforma Federal da Participação Social. Trata-se de mais
um espaço para participação social no Brasil, escuta e diálogo entre o Governo
Federal e a Sociedade Civil. 

A plataforma, totalmente desenvolvida em software livre, tem como missão
desenvolver práticas inovadoras de participação via internet e oferta de
espaços de manifestação e debate para qualquer cidadão ou organização, com o
intuito de construir políticas públicas cada vez mais eficazes e efetivas.

O Participa.br é desenvolvido sob a plataforma para redes sociais Noosfero.

\subsection{O Noosfero}

O Noosfero é uma plataforma web livre para redes sociais e de economia
solidária que possui as funcionalidades de Blog, e-Portfolios, CMS, RSS,
discussão temática, agenda de eventos e inteligência econômica colaborativa
num mesmo sistema! O Noosfero utiliza a linguagem de programação Ruby com
framework Rails e, portanto, suporta bancos de dados, PostgreSQL, MySQL,
SQLite entre outros.

Noosfero é um Software Livre e licenciado sob a GNU Affero General Public
License (AGPL), versão 3\cite{wikipediaSingleSignOn}.

\subsection{Aplicativos para trilhas de participação}
 
O Participa.br utiliza algumas ferramentas como mecanismo de participação,
essas ferramentas permitem aos usuários da rede contribuir através de
comentários, votos ou diretamente sugerindo textos e conteúdos \index{cms}.
%\newacronym{cms}{CMS}{Content Management System}

Estas ferramentas são organizadas em trilhas de participação, uma trilha
pode utilizar uma ou mais ferramentas e podem ter datas de de início e fim de
participação, dando assim um fluxo para a participação social \index{cms}

Nas próximas seções será apresentado 2 propostas de ferramentas ou aplicativos
para trilha de participação \index{cms}.

\subsubsection{Comentários por parágrafo}

Comentário por parágrafo o usuário pode comentar em parágrafos específicos de
um texto sugerindo alterações ou correções, ou mesmo tirando dúvidas.

\subsubsection{Pairwise}

Votação em pares...

\section{Conclusão}

Neste documento foi apresentado um \ProductDescription

Lembramos que para tornar o Portal de Consulta Pública realmente um canal de
consulta e participação popular na discussão e na definição da agenda
prioritária do país, é necessário que além de documentação faça-se um esforço
de movimentar as pessoar fora do ambiente virtual, para que haja um
engajamento no uso e contribuição deste projeto de forma consistente e perene.

\newpage
\bibliography{bibliografia}
\newpage
\listoffigures
\newpage
\printindex
\newpage
\section*{Anexos}
\begin{enumerate}
\item Classificação de conteúdo via mensagens etiquetadas e personas.
\item Seleção por rankeamento (tamanho de palavra) e limiar (número mínimo de palavras)
%\item Seleção por rankeamento de atividade dos usuários
\item Redes de amizade e de conteúdo do Participa.br
\begin{enumerate}
    \item Ordenação (ranking) por centralidade
    \item Setores conectivos
    \item Deteção de comunidades
\end{enumerate}
\item Exemplo em código computacional de classificação de conteúdo via conectividade dos participantes.
\item Script para testar o tempo de resposta do endpoint SparQL como conexão local e remota. OK
\item Experimentos online.
\end{enumerate}

%\appendix
%\appendixpage
%\input{appendix.tex}

\end{document}
